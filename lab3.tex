\documentclass[bachelor, och, labwork]{shiza}

\usepackage{subfigure}
\usepackage{tikz,pgfplots}
\pgfplotsset{compat=1.5}
\usepackage{float}

\usepackage{titlesec}
\setcounter{secnumdepth}{4}
\titleformat{\paragraph}
{\normalfont\normalsize}{\theparagraph}{1em}{}
\titlespacing*{\paragraph}
{35.5pt}{3.25ex plus 1ex minus .2ex}{1.5ex plus .2ex}

\titleformat{\paragraph}[block]
{\hspace{1.25cm}\normalfont}
{\theparagraph}{1ex}{}
\titlespacing{\paragraph}
{0cm}{2ex plus 1ex minus .2ex}{.4ex plus.2ex}

% --------------------------------------------------------------------------%


\usepackage[T2A]{fontenc}
\usepackage[utf8]{inputenc}
\usepackage{graphicx}
\graphicspath{ {./images/} }
\usepackage{tempora}

\usepackage[sort,compress]{cite}
\usepackage{amsmath}
\usepackage{amssymb}
\usepackage{amsthm}
\usepackage{fancyvrb}
\usepackage{listings}
\usepackage{listingsutf8}
\usepackage{longtable}
\usepackage{array}
\usepackage[english,russian]{babel}

\usepackage[colorlinks=true]{hyperref}
\usepackage{url}

\usepackage{underscore}
\usepackage{setspace}
\usepackage{indentfirst} 
\usepackage{mathtools}
\usepackage{amsfonts}
\usepackage{enumitem}
\usepackage{tikz}
\usepackage{minted}

\newcommand{\eqdef}{\stackrel {\rm def}{=}}
\newcommand{\specialcell}[2][c]{%
\begin{tabular}[#1]{@{}c@{}}#2\end{tabular}}

\renewcommand\theFancyVerbLine{\small\arabic{FancyVerbLine}}


\begin{document}

% \chair{Кафедра теоретических основ компьютерной безопасности и криптографии}

\title{Умножение разреженных полиномов}

\course{3}

\group{331}

\department{факультета КНиИТ}

\napravlenie{10.05.01 "--- Компьютерная безопасность}

\author{Никитина Арсения Владимировича}

\satitle{доцент, к.т.о.к.б.и.к}

\saname{А. Н. Гамова}

\date{2022}

\maketitle

%-------------------------------------------------------------------------------
\tableofcontents

\intro
В данной работе будут рассмотрены принципы алгоритмов вычисления умножения 
разреженных полинонов, оценки работы алгоритмов в наилучшем и наихудшем случаях,
а также программная реализация алгоритмов. 

Арифметичесике операции, такие как умножение, над целыми числами и полиномами
лучше изучать в совокупности, так как многие алгоритмы, работающие с целыми 
числами по существую совпадают с алгоритмами, работающими с полиномами от одной
переменной. Это верно не только для таких операций, как умножение и деление, но
такоэе и для более сложно осписываемых операций. Например, нахождение вычета
целого числа по модулю, задаваемому другим целым числом.


\section{Определение полинома}

Если $i$ - неотрицательное целое число то размер($i$)=$\log i + 1$. Если $p(x)$
- полином, то размер($p$)=$CT(p) + 1$, где $CT(p)$ - степень полинома $p$, то
есть наибольшая степень переменной $x$ с нулевым коэффициентом.

Над целыми числами и полиномами моно выполнять приближенное деление. Если 
$a ~\text{и}~ b$ - два целых числа и $b \not= 0$, то найдется единственная пара
целых чисел $q ~\text{и}~ r$, для которых


\section{Программная реализация алгоритма}


\section{Оценка работы алгоритма}


\conclusion


\end{document}